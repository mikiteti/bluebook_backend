%=========================================================
% Encoding, language, fonts (pdfLaTeX)
%=========================================================
\usepackage[T1]{fontenc}
\usepackage[utf8]{inputenc}
\usepackage[english]{babel}
% \usepackage{lmodern}
\usepackage{microtype}
\usepackage[most]{tcolorbox}


\usepackage{titling}

\pretitle{\begin{center}\LARGE\bfseries\color{ink}}
\posttitle{\par\end{center}\vspace{-0.4em}}


% If you ever switch to XeLaTeX or LuaLaTeX, comment the 4 lines
% above and UNcomment the block below and change compiler in Overleaf.
% (Use % at the start of each line, NOT /* ... */.)
%
% \usepackage{fontspec}
% \defaultfontfeatures{Ligatures=TeX}
% \setmainfont{Latin Modern Roman}
% \setsansfont{Latin Modern Sans}
% \setmonofont{Latin Modern Mono}

%=========================================================
% Page layout and spacing
%=========================================================
\usepackage[a4paper,
            margin=2.5cm,
            headheight=16pt,
            includeheadfoot]{geometry}

\usepackage{setspace}
\setstretch{1.05}

\usepackage{parskip}  % no indent, space between paragraphs
% If you prefer indents, comment the previous line and instead:
% \usepackage{indentfirst}
% \setlength{\parindent}{1.2em}
% \setlength{\parskip}{0pt}

%=========================================================
% Colors — “Warm paper + ink + accents”
% (balanced, not blue-dominant; good for physics solutions)
%=========================================================
\usepackage{xcolor}

% % Base (page + text)
% \definecolor{paper}{HTML}{FCF7EF}   % warm paper
% \definecolor{ink}{HTML}{1F2428}     % near-black
% \definecolor{rule}{HTML}{D6D0C7}    % borders/rules
%
% % Accents (use consistently)
% \definecolor{navy}{HTML}{2B3A67}    % deep slate-navy (links/refs)
% \definecolor{teal}{HTML}{0F766E}    % definitions/notes
% \definecolor{violet}{HTML}{6D28D9}  % lemmas/claims
% \definecolor{amber}{HTML}{B45309}   % important/warnings
% \definecolor{rose}{HTML}{BE123C}    % contradictions/errors
% \definecolor{green}{HTML}{15803D}   % final results

\definecolor{paper}{HTML}{FFFFFF}
\definecolor{ink}{HTML}{111827}
\definecolor{rule}{HTML}{D1D5DB}
\definecolor{navy}{HTML}{1E3A8A}
\definecolor{teal}{HTML}{0F766E}
\definecolor{violet}{HTML}{7C3AED}
\definecolor{amber}{HTML}{92400E}
\definecolor{rose}{HTML}{9F1239}
\definecolor{green}{HTML}{166534}

\pagecolor{paper}

%---------------------------------------------------------
% Map to your existing names so the rest of the preamble
% keeps working without changes
%---------------------------------------------------------
\colorlet{mybg}{paper}
\colorlet{mygray}{ink}
\colorlet{mylightgray}{rule}
\colorlet{myblue}{navy} % (kept only as a compatibility alias)

%=========================================================
% box (keep EXACT colors as requested)

\definecolor{answerframe}{HTML}{2F6FDB}
\definecolor{answerback}{HTML}{EAF2FF}

\newtcolorbox{answerbox}[1][]{
  enhanced,
  breakable,
  colback=answerback,
  colframe=answerframe,
  boxrule=0.8pt,
  arc=2pt,
  left=6pt,
  right=6pt,
  top=6pt,
  bottom=6pt,
  halign=center,   % horizontal centering
  valign=center,   % vertical centering (effective if you set a height)
  #1
}




%=========================================================
% Griffiths-style note-taking boxes (using your palette)
% Functions: definition, example, derivation, result, pitfall, note
%=========================================================

%=========================================================
% Theorem tcolorbox (preamble)
% Uses your palette: paper/ink/rule/navy/violet
% Requires: \usepackage[most]{tcolorbox}
%=========================================================
\tcbset{
  theoremstyle/.style={
    enhanced,
    breakable,
    coltext=ink,
    boxrule=0.7pt,
    arc=2pt,
    left=6pt,right=6pt,top=6pt,bottom=6pt,
    before skip=10pt, after skip=10pt,
    fonttitle=\bfseries,
    coltitle=ink,
    attach title to upper=\quad,
  }
}

\newtcbtheorem[number within=section]{theorembox}{Theorem}{
  theoremstyle,
  colframe=violet,
  colback=violet!5,
}{thm}



% Global defaults for all tcolorboxes below (NOT answerbox)
\tcbset{
  enhanced,
  breakable,
  coltext=ink,
  boxrule=0.7pt,
  arc=2pt,
  left=6pt,right=6pt,top=6pt,bottom=6pt,
  before skip=10pt, after skip=10pt
}

% Numbered theorem-like boxes (titles + clean frames)
\newtcbtheorem[number within=section]{definitionbox}{Definition}{
  colframe=teal,
  colback=teal!6,
  fonttitle=\bfseries,
  coltitle=ink,
  attach title to upper=\quad,
}{def}

\newtcbtheorem[number within=section]{examplebox}{Example}{
  colframe=violet,
  colback=violet!6,
  fonttitle=\bfseries,
  coltitle=ink,
  attach title to upper=\quad,
}{ex}

\newtcbtheorem[number within=section]{derivationbox}{Derivation}{
  colframe=navy,
  colback=navy!5,
  fonttitle=\bfseries,
  coltitle=ink,
  attach title to upper=\quad,
}{der}

\newtcbtheorem[number within=section]{resultbox}{Result}{
  colframe=green,
  colback=green!6,
  fonttitle=\bfseries,
  coltitle=ink,
  attach title to upper=\quad,
}{res}

\newtcbtheorem[number within=section]{pitfallbox}{Common pitfall}{
  colframe=amber,
  colback=amber!7,
  fonttitle=\bfseries,
  coltitle=ink,
  attach title to upper=\quad,
}{pit}

% Unnumbered quick note box
\newtcolorbox{notebox}[1][]{%
  enhanced,
  breakable,
  colframe=rule,
  colback=paper,
  coltext=ink,
  title=\textbf{Note},
  fonttitle=\bfseries,
  coltitle=ink,
  boxrule=0.7pt,
  arc=2pt,
  left=6pt,right=6pt,top=6pt,bottom=6pt,
  before skip=10pt, after skip=10pt,
  #1
}

% Convenience wrappers (short names you can type quickly)
% Usage examples:
%   \begin{definition}...\end{definition}
%   \begin{definition}[Gauss law]...\end{definition}
\newenvironment{definition}[1][]{\begin{definitionbox}{#1}{} }{\end{definitionbox}}
\newenvironment{example}[1][]{\begin{examplebox}{#1}{} }{\end{examplebox}}
\newenvironment{derivation}[1][]{\begin{derivationbox}{#1}{} }{\end{derivationbox}}
\newenvironment{result}[1][]{\begin{resultbox}{#1}{} }{\end{resultbox}}
\newenvironment{pitfall}[1][]{\begin{pitfallbox}{#1}{} }{\end{pitfallbox}}
\newenvironment{note}[1][]{\begin{notebox}[title=\textbf{Note}\if\relax\detokenize{#1}\relax\else\space(#1)\fi]}{\end{notebox}}

%=========================================================
% Mathematics
%=========================================================
\usepackage{amsmath,amssymb,amsthm,mathtools}
\usepackage{bm}
\usepackage{physics}
\usepackage{siunitx}
\sisetup{
  detect-all,
  per-mode=symbol,
  separate-uncertainty = true,
  exponent-product = \cdot
}
\usepackage{icomma}
\usepackage{cancel}

%=========================================================
% Theorem-like environments
%=========================================================
\theoremstyle{plain}
\newtheorem{theorem}{Theorem}[section]
\newtheorem{lemma}[theorem]{Lemma}
\newtheorem{proposition}[theorem]{Proposition}
\newtheorem{corollary}[theorem]{Corollary}

\theoremstyle{definition}
\newtheorem{definitionthm}[theorem]{Definition}
\newtheorem{examplethm}[theorem]{Example}

\theoremstyle{remark}
\newtheorem{remark}[theorem]{Remark}

%=========================================================
% Figures, tables, graphics
%=========================================================
\usepackage{graphicx}
\usepackage{caption}
\usepackage{subcaption}
\captionsetup{
  font=small,
  labelfont=bf,
  labelsep=period
}

\usepackage{booktabs}
\usepackage{multirow}
\usepackage{array}
\usepackage{tabularx}
\usepackage{longtable}

%=========================================================
% TikZ and drawing
%=========================================================
\usepackage{tikz}
\usetikzlibrary{
  calc,
  arrows.meta,
  decorations.pathmorphing,
  patterns,
  angles,
  quotes,
  positioning
}

%=========================================================
% Lists
%=========================================================
\usepackage{enumitem}
\setlist{
  topsep=2pt,
  itemsep=2pt,
  parsep=0pt
}

%=========================================================
% Hyperlinks and clever references
%=========================================================
\usepackage{hyperref}
\hypersetup{
  colorlinks = true,
  linkcolor  = navy,
  citecolor  = navy,
  urlcolor   = navy,
}

\usepackage[capitalise,noabbrev]{cleveref}

%=========================================================
% Quotations and bibliography (optional)
%=========================================================
\usepackage{csquotes}

% Comment out this whole block if you are not using biblatex/biber.
\usepackage[
  backend=biber,
  style=authoryear,
  sorting=nyt,
  giveninits=true,
  maxbibnames=6
]{biblatex}
\addbibresource{references.bib}

%=========================================================
% Header, footer, page style  (Page X of Y + top line)
%=========================================================
\usepackage{fancyhdr}
\usepackage{lastpage} % <-- add this

\pagestyle{fancy}
\fancyhf{}

\fancyhead[L]{\small\nouppercase{\leftmark}}
\fancyhead[R]{\small\nouppercase{\rightmark}}

% <-- change this line:
\fancyfoot[C]{\small Page \thepage\ of \pageref{LastPage}}

\renewcommand{\headrulewidth}{0.4pt}
\renewcommand{\headrule}{\hbox to\headwidth{%
  \color{mylightgray}\leaders\hrule height \headrulewidth\hfill}}

% <-- change these:
\renewcommand{\footrulewidth}{0.4pt}
\renewcommand{\footrule}{\hbox to\headwidth{%
  \color{mylightgray}\leaders\hrule height \footrulewidth\hfill}}


%=========================================================
% Section title formatting
%=========================================================
\usepackage{titlesec}

\titleformat{\section}
  {\large\bfseries\color{navy}}
  {\thesection}{0.75em}{}

\titleformat{\subsection}
  {\normalsize\bfseries\color{teal}}
  {\thesubsection}{0.75em}{}

\titleformat{\subsubsection}
  {\normalsize\bfseries\color{ink}}
  {\thesubsubsection}{0.75em}{}

\titlespacing*{\section}{0pt}{*3}{*1.5}
\titlespacing*{\subsection}{0pt}{*2}{*1}
\titlespacing*{\subsubsection}{0pt}{*1.5}{*0.75}

%=========================================================
% Elegant page border
%=========================================================
\usepackage{eso-pic}

\AddToShipoutPictureBG{%
  \begin{tikzpicture}[remember picture,overlay]
    \draw[mylightgray, line width=0.7pt]
      ($(current page.north west) + (1.2cm,-1.2cm)$)
      rectangle
      ($(current page.south east) + (-1.2cm,1.2cm)$);
  \end{tikzpicture}
}
